% Options for packages loaded elsewhere
\PassOptionsToPackage{unicode}{hyperref}
\PassOptionsToPackage{hyphens}{url}
%
\documentclass[
]{book}
\usepackage{amsmath,amssymb}
\usepackage{lmodern}
\usepackage{iftex}
\ifPDFTeX
  \usepackage[T1]{fontenc}
  \usepackage[utf8]{inputenc}
  \usepackage{textcomp} % provide euro and other symbols
\else % if luatex or xetex
  \usepackage{unicode-math}
  \defaultfontfeatures{Scale=MatchLowercase}
  \defaultfontfeatures[\rmfamily]{Ligatures=TeX,Scale=1}
\fi
% Use upquote if available, for straight quotes in verbatim environments
\IfFileExists{upquote.sty}{\usepackage{upquote}}{}
\IfFileExists{microtype.sty}{% use microtype if available
  \usepackage[]{microtype}
  \UseMicrotypeSet[protrusion]{basicmath} % disable protrusion for tt fonts
}{}
\makeatletter
\@ifundefined{KOMAClassName}{% if non-KOMA class
  \IfFileExists{parskip.sty}{%
    \usepackage{parskip}
  }{% else
    \setlength{\parindent}{0pt}
    \setlength{\parskip}{6pt plus 2pt minus 1pt}}
}{% if KOMA class
  \KOMAoptions{parskip=half}}
\makeatother
\usepackage{xcolor}
\IfFileExists{xurl.sty}{\usepackage{xurl}}{} % add URL line breaks if available
\IfFileExists{bookmark.sty}{\usepackage{bookmark}}{\usepackage{hyperref}}
\hypersetup{
  pdftitle={Introducción a R},
  pdfauthor={Diego J. Lizcano},
  hidelinks,
  pdfcreator={LaTeX via pandoc}}
\urlstyle{same} % disable monospaced font for URLs
\usepackage{color}
\usepackage{fancyvrb}
\newcommand{\VerbBar}{|}
\newcommand{\VERB}{\Verb[commandchars=\\\{\}]}
\DefineVerbatimEnvironment{Highlighting}{Verbatim}{commandchars=\\\{\}}
% Add ',fontsize=\small' for more characters per line
\usepackage{framed}
\definecolor{shadecolor}{RGB}{248,248,248}
\newenvironment{Shaded}{\begin{snugshade}}{\end{snugshade}}
\newcommand{\AlertTok}[1]{\textcolor[rgb]{0.94,0.16,0.16}{#1}}
\newcommand{\AnnotationTok}[1]{\textcolor[rgb]{0.56,0.35,0.01}{\textbf{\textit{#1}}}}
\newcommand{\AttributeTok}[1]{\textcolor[rgb]{0.77,0.63,0.00}{#1}}
\newcommand{\BaseNTok}[1]{\textcolor[rgb]{0.00,0.00,0.81}{#1}}
\newcommand{\BuiltInTok}[1]{#1}
\newcommand{\CharTok}[1]{\textcolor[rgb]{0.31,0.60,0.02}{#1}}
\newcommand{\CommentTok}[1]{\textcolor[rgb]{0.56,0.35,0.01}{\textit{#1}}}
\newcommand{\CommentVarTok}[1]{\textcolor[rgb]{0.56,0.35,0.01}{\textbf{\textit{#1}}}}
\newcommand{\ConstantTok}[1]{\textcolor[rgb]{0.00,0.00,0.00}{#1}}
\newcommand{\ControlFlowTok}[1]{\textcolor[rgb]{0.13,0.29,0.53}{\textbf{#1}}}
\newcommand{\DataTypeTok}[1]{\textcolor[rgb]{0.13,0.29,0.53}{#1}}
\newcommand{\DecValTok}[1]{\textcolor[rgb]{0.00,0.00,0.81}{#1}}
\newcommand{\DocumentationTok}[1]{\textcolor[rgb]{0.56,0.35,0.01}{\textbf{\textit{#1}}}}
\newcommand{\ErrorTok}[1]{\textcolor[rgb]{0.64,0.00,0.00}{\textbf{#1}}}
\newcommand{\ExtensionTok}[1]{#1}
\newcommand{\FloatTok}[1]{\textcolor[rgb]{0.00,0.00,0.81}{#1}}
\newcommand{\FunctionTok}[1]{\textcolor[rgb]{0.00,0.00,0.00}{#1}}
\newcommand{\ImportTok}[1]{#1}
\newcommand{\InformationTok}[1]{\textcolor[rgb]{0.56,0.35,0.01}{\textbf{\textit{#1}}}}
\newcommand{\KeywordTok}[1]{\textcolor[rgb]{0.13,0.29,0.53}{\textbf{#1}}}
\newcommand{\NormalTok}[1]{#1}
\newcommand{\OperatorTok}[1]{\textcolor[rgb]{0.81,0.36,0.00}{\textbf{#1}}}
\newcommand{\OtherTok}[1]{\textcolor[rgb]{0.56,0.35,0.01}{#1}}
\newcommand{\PreprocessorTok}[1]{\textcolor[rgb]{0.56,0.35,0.01}{\textit{#1}}}
\newcommand{\RegionMarkerTok}[1]{#1}
\newcommand{\SpecialCharTok}[1]{\textcolor[rgb]{0.00,0.00,0.00}{#1}}
\newcommand{\SpecialStringTok}[1]{\textcolor[rgb]{0.31,0.60,0.02}{#1}}
\newcommand{\StringTok}[1]{\textcolor[rgb]{0.31,0.60,0.02}{#1}}
\newcommand{\VariableTok}[1]{\textcolor[rgb]{0.00,0.00,0.00}{#1}}
\newcommand{\VerbatimStringTok}[1]{\textcolor[rgb]{0.31,0.60,0.02}{#1}}
\newcommand{\WarningTok}[1]{\textcolor[rgb]{0.56,0.35,0.01}{\textbf{\textit{#1}}}}
\usepackage{longtable,booktabs,array}
\usepackage{calc} % for calculating minipage widths
% Correct order of tables after \paragraph or \subparagraph
\usepackage{etoolbox}
\makeatletter
\patchcmd\longtable{\par}{\if@noskipsec\mbox{}\fi\par}{}{}
\makeatother
% Allow footnotes in longtable head/foot
\IfFileExists{footnotehyper.sty}{\usepackage{footnotehyper}}{\usepackage{footnote}}
\makesavenoteenv{longtable}
\usepackage{graphicx}
\makeatletter
\def\maxwidth{\ifdim\Gin@nat@width>\linewidth\linewidth\else\Gin@nat@width\fi}
\def\maxheight{\ifdim\Gin@nat@height>\textheight\textheight\else\Gin@nat@height\fi}
\makeatother
% Scale images if necessary, so that they will not overflow the page
% margins by default, and it is still possible to overwrite the defaults
% using explicit options in \includegraphics[width, height, ...]{}
\setkeys{Gin}{width=\maxwidth,height=\maxheight,keepaspectratio}
% Set default figure placement to htbp
\makeatletter
\def\fps@figure{htbp}
\makeatother
\setlength{\emergencystretch}{3em} % prevent overfull lines
\providecommand{\tightlist}{%
  \setlength{\itemsep}{0pt}\setlength{\parskip}{0pt}}
\setcounter{secnumdepth}{5}
\usepackage{booktabs}
\usepackage{amsthm}
\ifxetex
  \usepackage{polyglossia}
  \setmainlanguage{spanish}
  % Tabla en lugar de cuadro
  \gappto\captionsspanish{\renewcommand{\tablename}{Tabla}  
          \renewcommand{\listtablename}{Índice de tablas}}

\else
  \usepackage[spanish,es-tabla]{babel}
\fi
\makeatletter
\def\thm@space@setup{%
  \thm@preskip=8pt plus 2pt minus 4pt
  \thm@postskip=\thm@preskip
}
\makeatother
\ifLuaTeX
  \usepackage{selnolig}  % disable illegal ligatures
\fi
\usepackage[]{natbib}
\bibliographystyle{apalike}

\title{Introducción a R}
\author{Diego J. Lizcano}
\date{2022-07-08}

\begin{document}
\maketitle

{
\setcounter{tocdepth}{1}
\tableofcontents
}
\hypertarget{pruxf3logo}{%
\chapter*{Prólogo}\label{pruxf3logo}}
\addcontentsline{toc}{chapter}{Prólogo}

Este libro es una pequeña guía y tutorial sobre como emplear el lenguaje estadistico \texttt{R}

\begin{flushleft}\includegraphics[width=2.78in]{images/R} \end{flushleft}

Este libro ha sido escrito en \href{http://rmarkdown.rstudio.com}{R-Markdown} empleando el paquete \href{https://bookdown.org/yihui/bookdown/}{\texttt{bookdown}}

\begin{flushleft}\includegraphics[width=2.78in]{images/rmd} \end{flushleft}

Este libro está disponible en el repositorio Github: \href{https://github.com/dlizcano/IntroR}{dlizcano/IntroR}.

Esta obra está bajo una licencia de \href{https://creativecommons.org/licenses/by-sa/4.0/deed.es}{Creative Commons Reconocimiento-Compartir Igual 4.0 Internacional}.

\begin{flushleft}\includegraphics[width=1.22in]{images/by-sa-88x31} \end{flushleft}

\hypertarget{intro}{%
\chapter{Introducción}\label{intro}}

Placeholder

\hypertarget{requisitos}{%
\section{Requisitos}\label{requisitos}}

\hypertarget{primeros-pasos}{%
\section{Primeros pasos}\label{primeros-pasos}}

\hypertarget{creando-objetos-simples}{%
\chapter{Creando objetos simples}\label{creando-objetos-simples}}

Placeholder

\hypertarget{vectores}{%
\section{Vectores}\label{vectores}}

\hypertarget{operaciones-basicas-con-vectores}{%
\section{Operaciones basicas con vectores}\label{operaciones-basicas-con-vectores}}

\hypertarget{inspeccionando-vectores}{%
\section{Inspeccionando vectores}\label{inspeccionando-vectores}}

\hypertarget{matrices}{%
\chapter{Matrices}\label{matrices}}

Placeholder

\hypertarget{tablas}{%
\chapter{Tablas}\label{tablas}}

Placeholder

\hypertarget{crear-data-frames}{%
\section{Crear Data Frames}\label{crear-data-frames}}

\hypertarget{indexar-data-frames}{%
\section{Indexar data frames}\label{indexar-data-frames}}

\hypertarget{crear-una-columna-nueva}{%
\section{Crear una columna nueva}\label{crear-una-columna-nueva}}

\hypertarget{visualizar-la-tabla-graficamente}{%
\section{Visualizar la tabla graficamente}\label{visualizar-la-tabla-graficamente}}

\hypertarget{como-histograma}{%
\subsection{Como histograma}\label{como-histograma}}

\hypertarget{como-dos-variables-numericas}{%
\subsection{Como dos variables numericas}\label{como-dos-variables-numericas}}

\hypertarget{diagrama-de-caja-boxplots}{%
\subsection{Diagrama de caja (boxplots)}\label{diagrama-de-caja-boxplots}}

\hypertarget{creaciuxf3n-y-e-indexaciuxf3n-de-listas}{%
\chapter{Creación y e indexación de listas}\label{creaciuxf3n-y-e-indexaciuxf3n-de-listas}}

Placeholder

\hypertarget{creaciuxf3n-de-funciones}{%
\chapter{Creación de Funciones}\label{creaciuxf3n-de-funciones}}

Esta sección, por su importancia, pertenece propiamente a la sección de programación. La creación de funciones en lo que sigue del curso es fundemental y, presenta la gran versatilidad de un lenguaje de programación.

Creemos una función que opera sobre un vector que vamos a llamar x

\begin{Shaded}
\begin{Highlighting}[]
\NormalTok{media }\OtherTok{\textless{}{-}} \ControlFlowTok{function}\NormalTok{(x)\{     }\CommentTok{\# inicio de la funcion}
\NormalTok{ longitud }\OtherTok{\textless{}{-}} \FunctionTok{length}\NormalTok{(x)}
\NormalTok{ suma }\OtherTok{\textless{}{-}} \FunctionTok{sum}\NormalTok{(x)}
 \FunctionTok{return}\NormalTok{ (suma }\SpecialCharTok{/}\NormalTok{ longitud) }\CommentTok{\# devuelve el resultado de la media}
\NormalTok{\}                         }\CommentTok{\# final de la funcion}


\CommentTok{\# creemos un vector}
\NormalTok{vector1 }\OtherTok{\textless{}{-}} \FunctionTok{rnorm}\NormalTok{ (}\DecValTok{100}\NormalTok{) }\CommentTok{\# 100 datos al azar de la distribucion normal}

\CommentTok{\# apliquemos la funcion al vector}
\FunctionTok{media}\NormalTok{(vector1)}
\end{Highlighting}
\end{Shaded}

\begin{verbatim}
## [1] -0.03523709
\end{verbatim}

\hypertarget{creaciuxf3n-de-bucles}{%
\chapter{Creación de bucles}\label{creaciuxf3n-de-bucles}}

Placeholder

\hypertarget{convirtamolo-en-funciuxf3n}{%
\section{Convirtamolo en función}\label{convirtamolo-en-funciuxf3n}}

\hypertarget{modelos-en-r}{%
\chapter{Modelos en R}\label{modelos-en-r}}

Placeholder

\hypertarget{modelo-de-la-media}{%
\section{Modelo de la media}\label{modelo-de-la-media}}

\hypertarget{modelo-de-regresiuxf3n-lineal}{%
\chapter{Modelo de regresión lineal}\label{modelo-de-regresiuxf3n-lineal}}

Placeholder

\hypertarget{regresiuxf3n-lineal-simple}{%
\section{Regresión lineal simple}\label{regresiuxf3n-lineal-simple}}

\hypertarget{regresiuxf3n-lineal-con-2-o-muxe1s-predictores}{%
\section{Regresión lineal con 2 o más predictores}\label{regresiuxf3n-lineal-con-2-o-muxe1s-predictores}}

\hypertarget{modelos-de-distribuciuxf3n}{%
\chapter{Modelos de distribución}\label{modelos-de-distribuciuxf3n}}

Placeholder

\hypertarget{binomial}{%
\section{Binomial}\label{binomial}}

\hypertarget{poisson}{%
\section{Poisson}\label{poisson}}

\hypertarget{normal-gausiana}{%
\section{Normal (Gausiana)}\label{normal-gausiana}}

  \bibliography{book.bib,packages.bib}

\end{document}
